\begin{resumen}

El objeto principal de este Trabajo Fin de Grado es conocer en profundidad modelos de comportamiento colectivo, más concretamente, el modelo de Cucker Smale y dos modelos adicionales que añaden un control, el modelo propuesto por Cañizo y col. \cite{canizo2010collective} y el propuesto por Trèlat y col. \cite{caponigro2015sparse}. Cucker Smale propone un modelo que pretende describir el comportamiento colectivo en bandadas de pájaros, aportando un sistema de ecuaciones que responde y es capaz de simular ese comportamiento en función, principalmente, del llamado parámetro de convergencia, $\beta$. Los autores que intervienen en el modelo de Cañizo, proponen un modelo basado en el de Cucker Smale al cual añaden la influencia que ejercen unos individuos sobre otros en función del promedio de la velocidad relativa del grupo y Trèlat y col. centran sus estudios en una premisa muy clara que es reducir el gasto energético que conlleva controlar el grupo hasta conseguir el consenso, centrándose así en generar un impulso mayor sobre el individuo más desviado.

Todos los modelos anteriores se han implementado en Matlab de manera que fuera posible obtener datos que permitan comparar el comportamiento del grupo al regirse por unas reglas u otras. Tras estas simulaciones se ha podido mostrar por una parte que el modelo de Cañizo es el más rápido en llegar a un estado de consenso bajo condiciones que ayuden conseguirlo y el de Trèlat que impulsa sólo a un individuo consigue llegar al consenso con menor gasto energético, en un tiempo finito ligeramente mayor, y por otra parte que bajo condiciones que compliquen el consenso, es Trélat quien tiene más facilidades para conseguirlo.

\end{resumen}

\begin{abstract}
The main purpose of this undergraduate thesis is to know in depth the collective behavior models, specifically, the Cucker Smale model and two additional control models, the model proposed by Cañizo et al. \cite{canizo2010collective} and the one proposed by Trèlat et al. \cite{caponigro2015sparse}. Cucker Smale propose a model to describe the behavior of a flock of birds, providing a system of equations that is capable of simulating that behavior based mainly on the so-called convergence parameter, $ \beta $. The authors involved in the Cañizo model propose a model based on the Cucker Smale one to which they add the influence exerted by some individuals on others based on the average relative speed of the group, and Trèlat et al. focus their studies on a very clear premise, which is to get the most efficient way in controlling the group to achieve the consensus, thus focusing on generating a greater impulse on the most sparsed agent.

All the previous models have been tested through Matlab simulations, so that it was possible to compare the group behavior governed by the different models. After these simulations, it has been possible to show, on the one hand, that the Cañizo model is the fastest in reaching a state of consensus when the conditions aims to reach it and that of Trèlat, which control only one individual, manages to reach consensus with less energy expenditure, in a slightly longer time and on the other hand when the conditions make reaching consensus complicated, Trèlat is more efficient.
\end{abstract}
