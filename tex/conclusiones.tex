\chapter{Conclusiones y trabajos futuros} \label{ch:conclusiones}

\section{Conclusiones} \label{s5_1}
Estudiar patrones que se dan de forma natural en el entorno que nos rodea puede resultar un reto debido a la aparente aleatoriedad en todo lo que sucede pero a la vez que complejo, resulta interesante cuando se valora la opción de poder aplicar estos comportamientos de la naturaleza y de otros seres vivos en soluciones de nuestro día a día.

Gracias a una lectura exhaustiva de los estudios y trabajos más importantes que se han basado en estos patrones, se ha adquirido una visión general de qué comportamientos existen en el medio natural y de qué manera pueden ser útiles para nosotros. Tal y como se pretendía, como primeros objetivos (O1 y O2) cumplido se ha aprendido cómo y para qué se organizan algunos insectos tan conocidos como las hormigas o las abejas, o animales como los peces o las aves, distinguiendo dos maneras de organización colectiva, una atendiendo a los elementos individuales y otra con una visión mucho más general relativa al grupo en su conjunto. 

Como objeto principal de este trabajo tenemos conocer en profundidad el modelo de Cucker Smale y dos modelos adicionales que lo controlan, el modelo propuesto por Cañizo et al. \cite{canizo2010collective} y el propuesto por Trèlat et al. \cite{caponigro2015sparse}. En el capítulo \ref{cap3} se han entendido cada uno de ellos desde el aspecto teórico, conociendo pues, que Cucker Smale basa su modelo en el comportamiento colectivo de las aves, aportando un sistema de ecuaciones que responde y es capaz de simular ese comportamiento en función, principalmente, del llamado parámetro de convergencia, $\beta$. Es en un función de este parámetro, que una bandada de pájaros simulada, conseguiría o no llegar al consenso siendo para $\beta\leq0.5$ siempre apto para conseguirlo en un tiempo finito y para $\beta>0.5$ solamente apto para llegar al consenso en tiempo finito bajo una condición de posición y direcciones concretas aunque siempre podría hacerlo en un tiempo infinito (objetivo O3). Por otra parte, en CCR, los autores proponen un modelo de Cucker Smale al cual añaden la influencia que ejercen unos individuos sobre otros en función del promedio de la velocidad relativa del grupo demostrando que a parte de las condiciones del parámetro de consenso $\beta$ de Cucker Smale, con este modelo también se consigue el consenso y la agrupación de todos los individuos para $0<p<2$. El modelo de control de Trèlat, basa sus estudios en una premisa muy clara que es el gasto energético que conlleva controlar el grupo para conseguir el consenso, centrándose así en generar un impulso mayor de manera individual sobre individuo más desviado, en vez de hacerlo sobre todo el grupo a la vez. Para ello compara cada uno con la media de las velocidades del grupo. Con esto, puede darse como completado el objetivo O4.

Todos los modelos anteriores se han probado en Matlab de manera que fuera posible obtener datos sobre los que apoyar los resultados obtenidos. Por una parte se ha comparado el comportamiento del modelo de Cucker Smale en función del parámetro $\beta$ realizando ejecuciones con valores que favorezcan el consenso en un tiempo finito ($\beta=0.25$) así como con valores que no lo hagan ($\beta=0.75$) obteniendo que en el primer caso el grupo es capaz de llegar al consenso en un tiempo $t<20s$, al contrario que en el segundo caso (objetivo O6)

En el caso de las pruebas realizadas para contrastar el comportamiento del modelo de Cucker Smale, con los que le controlan (objetivo O5), se ha podido comprobar la efectividad de los diferentes controles, que permiten llegar al consenso de manera más rápida. Se han comparado para ellos 4 modelos diferentes: Cucker Smale, CCR, Trèlat impulsando a todos los individuos y Trèlat impulsando sólo al que presenta mayor diferencia con la media del grupo (objetivo 7 y 8) llegando a dos conclusiones principales, por una parte que el modelo de CCR es el más rápido en llegar a un estado de consenso y el de Trèlat que impulsa sólo a un individuo consigue llegar al consenso con menor gasto energético, en un tiempo finito ligeramente mayor. 

Por último se han hecho pruebas masivas en las que se pudiera comprobar que el consenso no depende de la posición inicial de una ejecución, si no que se cumple independientemente de la posición y dirección iniciales del grupo, simulando los 4 modelos con 500 estados iniciales diferentes, obteniendo que lo que se ha explicado en el párrafo anterior se cumple en la mayoría de los casos.


\section{Trabajos futuros} \label{s5_2}
Como se ha podido ver en el presente trabajo, los estudios relacionados con \textit{swarming} y el comportamiento colectivo tiene gran importancia, habiendo conseguido grandes avances en las últimas décadas. No obstante, seguir investigando y explorando nuevos modelos en la actualidad puede resultar de gran utilidad de cara a poder utilizarlos en aplicaciones cada vez más avanzadas. Se proponen varias lineas de trabajo futuros que según lo estudiado sin de gran interés:
\begin{itemize}
    \item En línea con trabajo actual, se propone añadir a las pruebas ya realizadas, el concepto de consenso del grupo propuesto por Cucker y Smale en el teorema 2 de su artículo \cite{cucker2007mathematics}, en vez del estado de consenso que se ha tomado en este trabajo como una diferencia máxima del vector velocidad menor que $0.01$. 
    \item También relacionado con este TFG, sería de gran interés realizar más pruebas utilizando el límite del parámetro de convergencia, $\beta = 0.5$ probando con diferentes estados iniciales de posición y velocidad, comparando el comportamiento con los otros modelos.
    \item Ejercer controles más avanzados en los que se añaden conceptos como el sorteo de obstáculos fijos y obstáculos móviles. Esto puede aplicarse para dotar a cada individuo de cierta inteligencia en cuanto a la detección de sus propios compañeros \cite{alejo2014optimal, berg2011reciprocal, van2008reciprocal}
    \item La clara relación con el mundo de la robótica y de los vehículos aéreos no tripulados (drones) permite pensar en una línea de investigación muy interesante con una gran cantidad de aplicaciones relacionadas con el control y la gestión aérea de problemas cotidianos. 
    
\end{itemize}
