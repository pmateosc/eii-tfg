\begin{resumen}

El objeto principal de este trabajo es conocer en profundidad el modelo de Cucker Smale y dos modelos adicionales que añaden un control, el modelo propuesto por Cañizo et al. \cite{canizo2010collective} y el propuesto por Trèlat et al. \cite{caponigro2015sparse}. Cucker Smale basa su modelo en el comportamiento colectivo de las aves, aportando un sistema de ecuaciones que responde y es capaz de simular ese comportamiento en función, principalmente, del llamado parámetro de convergencia, $\beta$. Los autores que intervienen en el modelo de Cañizo, proponen un modelo de Cucker Smale al cual añaden la influencia que ejercen unos individuos sobre otros en función del promedio de la velocidad relativa del grupo y Trèlat et al. centran sus estudios en una premisa muy clara que es reducir el gasto energético que conlleva controlar el grupo hasta conseguir el consenso, centrándose así en generar un impulso mayor sobre el individuo más desviado.

Todos los modelos anteriores se han probado en Matlab de manera que fuera posible obtener datos sobre los que apoyar los resultados obtenidos. Tras estas simulaciones se ha podido mostrar por una parte que el modelo de Cañizo es el más rápido en llegar a un estado de consenso y el de Trèlat que impulsa sólo a un individuo consigue llegar al consenso con menor gasto energético, en un tiempo finito ligeramente mayor. 
\end{resumen}

\begin{abstract}
The main purpose of this work is to know in depth the Cucker Smale model and two additional control models, the model proposed by Cañizo et al. \cite{canizo2010collective} and the one proposed by Trèlat et al. \cite{caponigro2015sparse}. Cucker Smale bases his model on the collective behavior of birds, providing a system of equations that is capable of simulating that behavior based mainly on the so-called convergence parameter, $ \ beta $. The authors involved in the Cañizo model propose a Cucker Smale model to which they add the influence exerted by some individuals on others based on the average relative speed of the group, and Trèlat et al. focus their studies on a very clear premise, which is to get the most ``economical'' way in controlling the group to achieve the consensus, thus focusing on generating a greater impulse on the most sparsed agent.

All the previous models have been tested in Matlab so that it was possible to obtain data on which to support the results obtained. After these simulations, it has been possible to show, on the one hand, that the Cañizo model is the fastest in reaching a state of consensus and that of Trèlat, which control only one individual, manages to reach consensus with less energy expenditure, in a slightly longer time.
\end{abstract}
